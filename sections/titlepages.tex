\pdfbookmark[0]{English title page}{label:titlepage_en}
\newgeometry{margin=2.5cm}
\aautitlepage{%
  \englishprojectinfo{
    Title på projetet %title
  }{%
    Projekt emnet  %theme
  }{%
    $7^{th}$ Semester, 2018 %project period
  }{%
    2119B % project group
  }{%
    %list of group members
    Andreas Leon Jorsal\\
    Regitze Degn Mikkelsen\\
    Vanessa Albrect\\
    Stefan Kjeldgaard\\
   }{%
    %list of supervisors
    Ann-Louise Andersen\\
    Maria Schiønning Larsen\\
  }{%
    1 % number of printed copies
  }{%
    $19^{th}$ December 2018 % date of completion
  }%
}{%department and address
  \textbf{School of Engineering and Science (SES)}\\
  Aalborg University\\
  \url{http://www.aau.dk}
}{% the abstract
This project has been made in collaboration with Desmi A/S, who are the producer of ballast water treatment systems, which the all international operating ships in the world has recently been obligated to install before 2025. Thus, Desmi has a challenge in shortening the time-to-market of their latest BWTS system, CompactClean. In order to reinforce that process, Desmi should have a clear strategy and be efficient in coordinating their internal efforts. However, upon further investigation, it turns out that Desmi seems to have issues in this regard. The group has been able to identify that the current strategy is missing some critical elements. Further, it is revealed that organisational issues might hamper the coordination and communication of strategic efforts. The solution that has been developed, is based on theory of Strategy Map and Balanced Scorecard, which seeks to accommodate and mitigate these issues. In conclusion, the project seems to be a starting point for Desmi in their journey of adapting a more systematic approach for strategy development.
}


\cleardoublepage

%